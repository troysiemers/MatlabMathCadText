{Resistance (in ohms) and current (in amps) are related through the equation I=V/R.  Data was collected from a circuit with unknown constant voltage and is shown in the table.
\[
\begin{array}{|c|c|}
\hline
{\bf Resistance (R) } & {\bf Current (I)} \\ \hline
10&	11.11\\ \hline
15&	8.04\\ \hline
25&	6.03\\ \hline
40&	2.77\\ \hline
65&	1.97\\ \hline
100&	1.51\\
\hline
\end{array}
\]
\begin{enumerate}
\item[a.] Plot R ($x$-axis) versus I ($y$-axis).  What kind of relationship do you see?
\item[b.] Plot 1/R ($x$-axis) versus I ($y$-axis).  Now, what kind of relationship do you see?
\item[c.] Use \tnr{intercept} and \tnr{slope} (or use \tnr{linfit}) to calculate the slope and intercept of the line in part {\bf b}.  
\item[d.] Approximate the current when the resistance is 80 ohms.  You can use the trace capability here.
\item[e.] Create a new plot with the data points from part b, a line with slope and intercept from part c, and the interpolated point (with a different symbol) from part d.  Label and title appropriately.
\end{enumerate}}
{}
%
%\item[a.] Plot R ($x$-axis) versus I ($y$-axis).  What kind of relationship do you see?
%\item[b.] Plot 1/R ($x$-axis) versus I ($y$-axis).  Now, what kind of relationship do you see?
%\item[c.] Use \cour{polyfit} to calculate the slope and intercept of the line in part {\bf b}.  What does the slope of the line represent?
%\item[d.] Use \cour{interp1} to approximate the current when the resistance is 80 ohms.
%\item[e.] Add a line with slope and intercept from part c and the interpolated point from part d to the plot in part b.  Label and title appropriately.