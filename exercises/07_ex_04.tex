{Modify your \cour{RemoveRow} Function from the previous homework in two ways (you should have two functions, perhaps called \cour{RemoveRowIf} and \cour{RemoveRowWhile})
\begin{enumerate}
\item[a.] In the first function, use an \cour{if} statement to have an error message displayed if the user enters an invalid row.  Then prompt them to enter a new row number.  In your prompt, you must indicate the allowable range of rows.  Here they only have one chance to re-enter the row.
\item[b.] In the second function, use a \cour{while} loop so that if the user enters an invalid row on their first try, they will be able to continue to enter rows until they enter a correct value.  Again, in your prompt, you must indicate the allowable range of rows. 
\end{enumerate}}
{}



%{In this exercise, you will be comparing the capabilities of the if$\backslash$else and switch$\backslash$case structures by creating two programs.  In each program, prompt the user to enter the last name of one of the last 10 presidents of the United States.  Your program should display an informative sentence in response (including the presidency number and when he served), or display an error message if the entered name is invalid.  In the first program, use an if$\backslash$else structure and in the second program, use a switch$\backslash$case structure.  }
%{}