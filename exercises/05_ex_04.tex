{Create a function called \cour{RemRow} with two inputs \cour{M} and \cour{n}.  The output should be a matrix \cour{NewM} that comes from removing the nth row of \cour{M}.}
{}
%that removes a row from a matrix.\\
%Details:\\
%The function name should be something appropriate: like \cour{siemerstjRemRow}\\
%There should be two input\\
%\cour{n -} the number of the row to be removed\\
%\cour{M -}  the matrix of which you want the row to be removed.  \\
%There should be one output\\
%\cour{NewM -}  the matrix that comes from removing the nth row of  \cour{M}.\\
%Example:  For\\
%\cour{>> M=[1 2 3; 4 5 6; 7 8 9]; }\\
%\cour{>> N=siemerstjRemRow(M,2)}\\
%�\\
%Result: (note that the 2nd row has been removed).\\
%�
%\cour{N =}\\
%\cour{\ps 1     2     3}\\
%\cour{\ps 7     8     9�}\\}
%{}%
