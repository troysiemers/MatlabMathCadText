{A uniform beam is freely hinged at its ends $x = 0$ and $x = L$, so that the ends are at the same level.  It carries a uniformly distributed load of $W$ per unit length and there is a tension $T$ along the $x$-axis.  The deflection $y$ of the beam a distance $x$ from on end is given by
\[
y = \frac{W \cdot EI}{T^2}\left[ \frac{\cosh\left[ a (L/2 - x)\right]}{\cosh(a\, L/2)} - 1 \right] 
+
\frac{W x(L-x)}{2T}
\]
where $a^2= T/EI$ , $E$ is Young's Modulus of the beam and $I$ is the moment of inertia of a cross-section of a beam.  If the beam is $10 m$ long, the tension is $1000N$, the load $100 N/m$ and $EI$ is $10^4$, make a table of $x$ versus $y$ where $x$ ranges from 0 to 10 in increments of $1 m$.  Make sure to define the variables $a, W, EI, T$ and $x$ and then define $y$ in terms of them.}
{}