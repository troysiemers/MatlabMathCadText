{In computing an approximating function for a set of data, a spline is often used.  The theory of splines is covered in numerical analysis, but for a specific data set, the following system of equations must be solved.
\[
\left\{
\begin{array}{ccccccccccc}
0.28 S_1 &+& 0.1 S_2 &&&&&&&&= -64.65\\
0.1 S_1 &+& 0.34 S_2 &+ &0.07 S_3 &&&&&&= -54.81\\
&&0.07 S_2 &+& 2.16 S_3 &+ &1.01 S_4 &&&&= - 8.43\\
&&&&1.01 S_3 &+ &2.58 S_4 &+& 0.28 S_5 &&= -7.92\\
&&&&&&0.28 S_4 &+ &1.42 S_5 &&= -2.78
\end{array}
\right\}
\]

Find the values for $S_1$ through $S_5$ by solving this system in three ways:
\begin{enumerate}
\item[a.] Using the command \tnr{rref}.
\item[b.] Using the command \tnr{lsolve}.
\item[c.] Using the matrix inverse.
\end{enumerate}}
{}