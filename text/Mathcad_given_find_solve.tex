Here we investigate using ``given/find'' and ``solve'' blocks for solving equations or systems of equations.


\section{Mathcad: Given/Find Blocks}\label{sec:Mathcadgivenfind}

\index{Mathcad Functions!\tnr{given$\backslash$find} block}
\noindent \large \textsf{\textbf{given/find block}} \normalsize\\

We illustrate this through the following example.\\

\example{exgivenfind1}{{\bf  Solve $2 = x^4+3x^2-1$.}\\

In the worksheet, the syntax is \\
\\
\tnr{given}\\
\tnr{2 $\boldeq$ x$^4$ + 3x$^2$ - 1}  \ps   	(Use ctrl + = for bold symbolic equals sign)\\
\tnr{guess}\\
\tnr{x := 1}	\\
\tnr{x := find(x)}  \\
\tnr{x =}	\ps \\
\\
The last line will show \tnr{x = 0.89} once the equals sign is entered.  The line \tnr{x := find(x)} solves for \tnr{x} and then stores it back as the variable \tnr{x}.  For full details, the guess of \tnr{x := 1} provides the ``seed'' for the Newton-Raphson algorithm (Google it).   Try changing the seed value to see what happens.  For example, if we start with \tnr{x := $-$1} we end up with a different answer, namely \tnr{x = $-$0.89}.
}\\

The given/find process can be extended to solving systems of equations as well as in the next example:\\

\example{ex_givenfind2}{{\bf Simultaneously solve $a + b = 3, a - b = 4$ for $a,b$.}\\

In the worksheet, using ctrl + = for bold symbolic equals sign, the syntax is\\
\\
\tnr{given}\\
\tnr{a+b $\boldeq$ 3}\\
\tnr{a-b $\boldeq$ 4}\\
\tnr{guess}\\
\tnr{a := 1}	\\
\tnr{b := 1}	\\
\tnr{find(a,b)=}  \\
\\
Once, you hit return on the last line, it will become\\
\\
\tnr{find(a,b)=}
$
\left(
\begin{array}{c}
3.5\\
-0.5
\end{array}
\right)
$\\
\\
So, the solution is $a=3.5, b=-0.5$.}\\

The given/find capabilities are not limited to linear equations as can be seen in the next example.\\

\example{ex_givenfind3}{ {\bf Simultaneously solve $t^4+r^3 = 1$ and $r - t^2 = -2$}
\\

In the worksheet, using ctrl + = for bold symbolic equals sign, the syntax is\\
\\
\tnr{given}\\
\tnr{t$^4$ + r$^3$ $\boldeq$ 1}\\
\tnr{r $-$ t$^2$ $\boldeq$ $-$2}\\
\tnr{guess}\\
\tnr{r := 1}	\\
\tnr{t := 1}	\\
\tnr{find(r,t)=}  \\
\\
Once, you hit return on the last line, it will become\\
\\
\tnr{find(r,t)=}
$
\left(
\begin{array}{c}
-0.783\\
1.103
\end{array}
\right)
$\\
\\
So, one solution is $r=-0.783, t=1.103$.  Note that this is not the only solution.  Can you change the seed (or ``guess'') values to get the rest of the solutions? 
}

\section{Mathcad: Solve Blocks}\label{sec:Mathcadsolve}

\index{Mathcad Functions!\tnr{solve} block}
\noindent \large \textsf{\textbf{solve block}} \normalsize\\  

To use the solve block, first open the symbolic toolbar.  We show how to solve the example \ref{exgivenfind1} with this new technique.\\

\example{ex_solveblock1}{ {\bf Solve $2 = x^4+3x^2-1$.}\\

In the worksheet, using ctrl + = for the bold symbolic equals sign and clicking on the word ``solve'' in the symbolic toolbar, the syntax is\\
\\
\tnr{2 $\boldeq$ x$^4$ + 3x$^2$ - 1 solve, x} $\rightarrow$\\
\\
Once you hit return, it will look like:\\
\\
\tnr{2 $\boldeq$ x$^4$ + 3x$^2$ - 1 solve, x} $\rightarrow$
$
\left(
\begin{array}{c}
\displaystyle -\sqrt{-\frac{1}{2}\sqrt{21}-\frac{3}{2}}\\
\displaystyle \sqrt{-\frac{\sqrt{21}}{2}-\frac{3}{2}}\\
\displaystyle -\sqrt{-\frac{1}{2}\sqrt{21}-\frac{3}{2}}\\
\displaystyle \sqrt{\frac{\sqrt{21}}{2}-\frac{3}{2}}\\
\end{array}
\right)
$

If you type an equals sign on the end of the last line, it will become:\\
\\
\tnr{2 $\boldeq$ x$^4$ + 3x$^2$ - 1 solve, x} $\rightarrow$
$
\left(
\begin{array}{c}
\displaystyle -\sqrt{-\frac{1}{2}\sqrt{21}-\frac{3}{2}}\\
\displaystyle \sqrt{-\frac{\sqrt{21}}{2}-\frac{3}{2}}\\
\displaystyle -\sqrt{-\frac{1}{2}\sqrt{21}-\frac{3}{2}}\\
\displaystyle \sqrt{\frac{\sqrt{21}}{2}-\frac{3}{2}}\\
\end{array}
\right)
=
\left(
\begin{array}{c}
-1.947i\\
1.947i\\
-0.89\\
0.89
\end{array}
\right)$
\\
Note that two of these solutions are not real valued.  Also, it is proper syntax to leave off the ``, x'' after the word solve as Mathcad will solve for the only variable present by default.  That is, \\ 
\\
\tnr{2 $\boldeq$ x$^4$ + 3x$^2$ - 1 solve} $\rightarrow$\\
\\
will still give the same solution.
}\\

The solve block can be extended to equations with multiple variables as seen in the next example.

\example{ex_solveblock2}{{\bf Solve $1 = \displaystyle\frac{x^2 y}{x^2 + y^2}$ for each variable}\\
\\
Using the bold equals and the word ``solve'' from the symbolic toolbar, the syntax is:\\
\\
\tnr{1 $\boldeq$ $\displaystyle \frac{\tnr{x}^2 \tnr{y}}{\tnr{x}^2 + \tnr{y}^2}$ solve, x} $\rightarrow$\\
\\
Once you hit return, it will look like:\\
\tnr{1 $\boldeq$ $\displaystyle \frac{\tnr{x}^2 \tnr{y}}{\tnr{x}^2 + \tnr{y}^2}$ solve, x} 
$\rightarrow 
\left(
\begin{array}{c}
\displaystyle -\frac{\tnr{y}}{\sqrt{\tnr{y}-1}}\\
\\
\displaystyle \frac{\tnr{y}}{\sqrt{\tnr{y}-1}}
\end{array}
\right)$\\
\\
and\\
\\
\tnr{1 $\boldeq$ $\displaystyle \frac{\tnr{x}^2 \tnr{y}}{\tnr{x}^2 + \tnr{y}^2}$ solve, y} $\rightarrow$\\
\\
Once you hit return, it will look like:\\
\tnr{1 $\boldeq$ $\displaystyle \frac{\tnr{x}^2 \tnr{y}}{\tnr{x}^2 + \tnr{y}^2}$ solve, y} 
$\rightarrow 
\left(
\begin{array}{c}
\displaystyle \frac{\tnr{x}(\tnr{x}+\sqrt{\tnr{x}^2-4})}{2}\\
\displaystyle \frac{\tnr{x}^2}{2} - \frac{\tnr{x}\sqrt{\tnr{x}^2-4}}{2}
\end{array}
\right)$\\
\po
}

\newpage
\printexercises{exercises/12_exercises}