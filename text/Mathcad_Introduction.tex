In this chapter, we discuss the basics of Mathcad.

\section{Mathcad: Introduction}\label{sec:Mathcad_introduction}

Let's dive right in and take a look at Mathcad. The first thing you notice that the screen looks like a blank sheet.  In fact, the vertical lines define the edges of how the document would be printed.

To enter information in a certain place, you can (left) click there and start typing (try it!).

Mathcad provides for a nice blend of text and mathematics, including equations, data and graphs.  There is also a good balance between using toolbar buttons and the flexibility of entering commands, for those who like ``command line'' type languages.

In general, Mathcad is inexpensive, easy to learn \& use and provides for readable documents.  Mathcad handles units and unit conversions very well with a large, built-in list of units.  There are many reference tables available including: Basic Science constants, Calculus Formula, Geometric tables, Mechanics, Electromagnetics, and Properties of Liquids, Solids, Gases \& Metals.  Mathcad has both the ability to combine numerical and symbolic capabilities. 

As for Mathcad's weaknesses, programming is awkward.  Also, while it can combine both the ability to do numerics and symbolics, there are packages that do these better individually, like Matlab for numerical calculations and Mathematica for symbolics.\\
\\
HOMEWORK ASSIGNMENT RULE:\\

When you turn in your assignments, you must provide a header file.  To create one, simply click near the top left corner of the page and start typing; you will be put into text mode as you type the first word and a space.

An example for the first homework may look like:\\
\\
Name: Troy Siemers \\
Assignment: Mathcad Chapter 10 and 11\\
Course: MA110 \\
Date: $<$fill in$>$\\
Description: In this file we investigate the overall layout of Mathcad and how the toolbars are used.\\

We refer you to the help files for additional information.  There are helpful ``Quick sheets'' that give tutorials on many subjects.
%
%\definition{def:lineintegral}{\textbf{TITLE}}
%
%\keyidea{idea:lineintprops}{\textbf{TITLE}}
%
%\printexercises{exercises/mathcad_introduction_exercises}
